\documentclass{beamer}
\usepackage{xeCJK}
\usepackage[xjtu,zh]{collegeBeamer}
\usepackage{booktabs}

\setCJKmainfont[
    Path = C:/Users/10451/AppData/Local/Microsoft/Windows/Fonts/,
    BoldFont = SourceHanSerifSC-Bold.otf,
]{SourceHanSerifSC-Regular.otf}
% when using Chinese, uncomment the following line
% \usepackage{xeCJK}

% meta-data
\title{金属电子逸出功虚拟仿真实验}
\subtitle{大学物理实验大作业报告}
\author{吉致贤\\梅现\\吴振豪}
\date{报告日期:到时候再说}

% document body
\begin{document}
    
    \maketitle

    \begin{frame}
    本次汇报主要介绍测量金属逸出功和磁控法测量电子荷质比,及两个实验的实验原理,以及展示我们的实验仿真软件。

    我们利用matlab GUI搭建仿真实验平台,平台共包含三个部分:实验测量界面、金属逸出功数据处理界面和电子荷质比数据处理界面。
    
    实验测量界面中,用户可来回切换两个实验,用户可调控灯丝电流、励磁电流和阳极电压,得到阳极电流和灯丝温度,同时用户还可观察到电子的运动过程。用户可将测量数据输入到数据处理界面,得到金属逸出功和电子荷质比的实验值,并附有数据图像展示。
    \end{frame}

    \section{实验原理}
    \subsection{金属逸出功原理}
    \begin{frame}{理查森-杜西曼公式}{\thesubsection \, \subsecname}
        \begin{columns}[T]
            \begin{column}{0.5\textwidth}
                如图所示,用钨丝作阴极的理想二极管,通以电流加热,并在阳极和阴极间加上正向电压时,在外电路中就有电流通过。电流的大小主要与灯丝温度及金属逸出功的大小有关。
                
                根据费米-狄拉克分布可以导出热电子发射遵守的\textbf{理查森-杜西曼公式}:
                \begin{equation}
                    I = AST^2\exp(-\frac{e\varphi}{kT})
                \end{equation}
            \end{column}
            \begin{column}{0.3\textwidth}
                \includegraphics[scale=0.2]{gallery/pic1.jpg}
                
            \end{column}
        \end{columns}
    \end{frame}

    \begin{frame}{理查森直线法}{\thesubsection \, \subsecname}
        从上式可知,只要测出I、A、S、T的值,就可以计算出阴极材料的电子逸出功,但是直接测定 A、S 这两个量比较困难。故我们采用理查森直线法,它可以避开 A、S 的测量,即不必求出 A、S 的具体数值,直接由发射电流 I 和灯丝温度 T 确定逸出功的值。

        由式(1)得到式(2):
        \begin{equation}
            \frac{I}{T^2} = AS\exp(-\frac{e\varphi}{kT})
        \end{equation}

        两边取常用对数得到:
        \begin{equation}
            \lg\frac{I}{T^2} = \lg AS - \frac{e\varphi}{2.303k}\cdot \frac{1}{T}
        \end{equation}
    \end{frame}

    \begin{frame}{灯丝温度}{\thesubsection \, \subsecname}
        灯丝温度对发射电流I的影响很大,因此准确测量灯丝温度对于减小测量误差非常重要。灯丝温度一般取2000K左右,常用光学高温计进行测量。若不测量灯丝温度,可以根据灯丝真实温度与灯丝电流的关系,由灯丝电流确定灯丝温度,钨丝的真实温度与加热电流的对应关系如下表所示。
        \begin{table}
            \footnotesize
            \caption{钨丝的真实温度与加热电流的对应关系}
            \centering
            \begin{tabular}{cccccccccc}
            \toprule
            \textbf{灯丝电流}/A & 0.600	& 0.625	& 0.650	& 0.675	& 0.700	& 0.725	& 0.750	& 0.775	& 0.800\\
            \midrule
            \textbf{灯丝温度}/(103K) & 1.88	& 1.92	& 1.96	& 2.00	& 2.04	& 2.08	& 2.12	& 2.16	& 2.20\\
            \bottomrule
            \end{tabular}
        \end{table}
    \end{frame}

    \begin{frame}{金属逸出功原理}{\thesubsection \, \subsecname}
        要使阴极发射的热电子连续不断地飞向阳极,形成阳极电流$I_a$,就必须在阳极与阴极之间外加一个加速电场$E_a$。但$E_a$的存在相当于使新的势垒高度比无外电场时降低了,这导致更多的电子逸出金属,因而使发射电流增大,这种外电场产生的电子发射效应称为\textbf{肖特基效应}。
        
        阴极发射电流$I_a$与阴极表面加速电场$E_a$的关系是式(4):
        \begin{equation}
            I_a = I \exp(\frac{\sqrt{e^3E_a}}{kT})
        \end{equation}
        式中,$I_a$和$I$分别表示加速电场为$E_a$和零时的发射电流。
    \end{frame}

    \begin{frame}{金属逸出功原理}{\thesubsection \, \subsecname}
        为了方便,一般将阴极和阳极制成共轴圆柱体,在忽略接触电势差等影响的条件下,阴极表面附近加速电场的场强为式(5)
        \begin{equation}
            E_a = \frac{U_a}{r_1\ln(\frac{r_2}{r_1})}
        \end{equation}
        式中,$r_1$、$r_2$分别为阴极及阳极圆柱面的半径,$U_a$为加速电压。

        将式(4)代入式(5)取对数得
        \begin{equation}
            \lg I_a = \lg I + \frac{\sqrt{e^3}}{2.303kT}\cdot \sqrt{\frac{U_a}{r_1\ln(\frac{r_2}{r_1})}}
        \end{equation}
    \end{frame}

    \begin{frame}{金属逸出功原理}{\thesubsection \, \subsecname}
        式(6)表明,对于一定尺寸的直热式真空二极管,$r_1$、$r_2$一定,在阴极的温度$T$一定时,$\lg I_a$与$\sqrt{U_a}$也成线性关系,$\lg I_a$-根号$\sqrt{U_a}$直线的延长线与纵轴的交点,即截距为$\lg I$,由此即可得到在一定温度下,加速电场为零时的热电子发射(饱和)电流$I$,这样就可消除$E_a$对发射电流的影响。
    
        综上所述,要测定某金属材料的逸出功,可将该材料制成理想二极管的阴极,测定阴极温度$T$、阳极电压$U_a$和发射电流$I_a$,用作图法得到零场电流$I$后,即可求出逸出功或逸出电势。
    \end{frame}

    \subsection{磁控法测电子荷质比原理}

    \begin{frame}{磁控法测电子荷质比原理}{\thesubsection \, \subsecname}
        在理想二极管中,阴极和阳极为一同轴圆柱系统。当阳极加有正电压时,从阴极发射的电子流受电场的作用将做径向运动,如图a。如果在理想二极管外面套一个通电励磁线圈,则原来沿径向运动的电子在轴向磁场作用下,运动轨迹将发生弯曲,如图b所示。
        \begin{figure}[htbp]
            \centering
            \includegraphics[scale=0.7]{gallery/pic2.jpg}
        \end{figure}
    \end{frame}

    \begin{frame}{磁控法测电子荷质比原理}{\thesubsection \, \subsecname}
        若进一步加强磁场使电子流运动如图c所示,这时电子运动到阳极附近,电子所受到的洛伦兹力减去电场力后的合力恰好等于电子沿阳极内壁圆周运动的向心力,因此电子流运动的轨迹也将沿阳极内壁做圆周运动,此时称为"临界状态"。若进一步增强磁场,电子运动的圆半径就会减小,以致电子根本无法靠拢阳极,就会造成阳极电流“断流”,如图d所示。
        \begin{figure}[htbp]
            \centering
            \includegraphics[scale=0.7]{gallery/pic2.jpg}
        \end{figure}
    \end{frame}

    \begin{frame}{磁控法测电子荷质比原理}{\thesubsection \, \subsecname}
        在一定的阳极电压下,阳极电流$I_a$与励磁电流$I_r$的关系,在开始阶段几乎不发生改变,随着励磁电流$I_r$的逐渐增加,$I_a$的变化曲率达到最大.之后,随着$I_r$的加大,$I_a$逐步减小,平降到0。在$I_a$-$I_r$曲线上取阳极电流最大值$I_a$约1/4高度的点作为阳极电流变化的临界点$Q$,临界点$Q$只是个统计的概念,实际上不同速率运动的电子的临界点是不同的。
    \end{frame}

    \begin{frame}{磁控法测电子荷质比原理}{\thesubsection \, \subsecname}
        在单电子近似的情况下,从阴极发射出的、质量为$m$的电子动能应有阳极加速电场能$eU_a$和灯丝加热后电子“热运动”所具能量$W$两部分构成,所以有式(7)
        \begin{equation}
            \frac{1}{2}mv^2 = eU_a + W
        \end{equation}
        电子在磁场$B$的作用下做半径为$R$的圆周运动,应满足式(8)
        \begin{equation}
            eBv = \frac{mv}{R}
        \end{equation}
        通电励磁线圈中心处的磁感应强度有式(9)
        \begin{equation}
            B = \frac{\mu_0NI_s}{2(r_2-r_1)}\ln \frac{r_2+\sqrt{r_2^2+L^2}}{r_1+\sqrt{r_1^2+L^2}} = K'I_s
        \end{equation}
        其中,$r_1$为线圈的内半径;$r_2$为线圈的外半径;$L$为线圈半长度;$K'$为比例分数。
    \end{frame}

    \begin{frame}{磁控法测电子荷质比原理}{\thesubsection \, \subsecname}
        由式(7)、(8)、(9)可得式(10)
        \begin{equation}
            \frac{U_a+W/e}{I_s^2} = \frac{e}{m} \cdot \frac{R^2}{2}\cdot K'^2
        \end{equation}
        若设阳极内半径为$a$,而阴极(灯丝)半径忽略不计,则当多数电子都处于临界状态,与临界点$Q$对应的励磁线圈的电流$I_a$称为临界电流$I_c$,而此时$R = a/2$,阳极电压$U_a$与$I_c$的关系可写为式(11)
        \begin{equation}
            \frac{U_a+W/e}{I_c^2} = \frac{e}{m} \cdot \frac{a}{8}\cdot K'^2 = K
        \end{equation}
    \end{frame}

    \begin{frame}{总结}{\thesubsection \, \subsecname}
        用同一个理想二极管,在不同的$U_a$下,就有不同的阳极电流与励磁电流变化曲线,因而就有不同的$I_c$值与之对应。再将测得的$U_a$-$I_c^2$数据组用图解法或最小二乘法球的斜率$K$,从而求得电子荷质比。
    \end{frame}

    \section{仿真软件展示}
    \begin{frame}{测量页面}{\thesection \, \secname}
        \begin{columns}[T]
            \begin{column}{0.4\textwidth}
                下面是我们仿真实验平台的展示:

                平台共包含三个部分,实验测量页面,金属逸出功的数据处理页面和电子荷质比数据处理页面,通过点击最上方按钮切换三种页面。
            \end{column}
            \begin{column}{0.3\textwidth}
                \includegraphics[scale=0.22]{gallery/pic3.jpg}
            \end{column}
        \end{columns}
        
    \end{frame}

    \begin{frame}{测量页面}{\thesection \, \secname}
        \begin{columns}
            \begin{column}{0.3\textwidth}
                \includegraphics[scale=0.3]{gallery/move1.jpg}
                \includegraphics[scale=0.3]{gallery/move2.jpg}
            \end{column}
            \begin{column}{0.3\textwidth}
                \includegraphics[scale=0.3]{gallery/move3.jpg}
                \includegraphics[scale=0.3]{gallery/move4.jpg}
            \end{column}
        \end{columns}
    \end{frame}

    \begin{frame}{测量页面}{\thesection \, \secname}
        \begin{columns}
            \begin{column}{0.3\textwidth}
                \includegraphics[scale=0.3]{gallery/move5.jpg}
                \includegraphics[scale=0.3]{gallery/move6.jpg}
            \end{column}
            \begin{column}{0.3\textwidth}
                \includegraphics[scale=0.3]{gallery/move7.jpg}
                \includegraphics[scale=0.3]{gallery/move8.jpg}
            \end{column}
        \end{columns}
    \end{frame}

    \begin{frame}{数据处理页面(逸出功)}{\thesection \, \secname}
        第二个界面为金属逸出功的数据处理页面:
        \begin{figure}
            \centering
            \includegraphics[scale=0.28]{gallery/pic4.jpg}
        \end{figure}
    \end{frame}
    \begin{frame}{数据处理页面(逸出功)}{\thesection \, \secname}
    用户可以点击“数据导入”按钮,导入实验数据,也可以在此基础上手动修改数据。

    点击“开始计算”按钮后,平台将会计算金属逸出功的实验值,给出与理论值的相对误差,并且采用最小二乘法,绘制出$\lg I_a$-$\sqrt{U}$图像和$\lg \frac{I}{T^2}$-$\frac{1}{T}$图像,方便用户检查数据。

        \begin{columns}
            \begin{column}{0.4\textwidth}
                \includegraphics[scale=0.27]{gallery/pic5.jpg}
            \end{column}
            \begin{column}{0.4\textwidth}
                \includegraphics[scale=0.8]{gallery/pic6.jpg}
            \end{column}
        \end{columns}
    \end{frame}

    \begin{frame}{数据处理页面(荷质比)}{\thesection \, \secname}
        第三个页面为电子荷质比的数据处理页面:
        \begin{figure}
            \centering
            \includegraphics[scale=0.28]{gallery/pic7.jpg}
        \end{figure}
    \end{frame}
    \begin{frame}{数据处理页面(荷质比)}{\thesection \, \secname}
    用户可以点击“数据导入”按钮,导入实验数据,也可以在此基础上手动修改数据。

    点击“开始计算”按钮后,平台将会计算电子荷质比的实验值,给出与理论值的相对误差,并且绘制出$I_a$-$I_s$图像,取最大值的1/4处为临界电流$I_c$,并绘制阳极电压与临界电流平方$U_a$-$I_c^2$图,方便用户检查数据。

        \begin{columns}
            \begin{column}{0.4\textwidth}
                \includegraphics[scale=0.27]{gallery/pic8.jpg}
            \end{column}
            \begin{column}{0.4\textwidth}
                \includegraphics[scale=0.8]{gallery/pic9.jpg}
            \end{column}
        \end{columns}
    \end{frame}
\end{document}